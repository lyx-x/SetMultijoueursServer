%----------------------------------------------------------------------------------------
%	PACKAGES AND OTHER DOCUMENT CONFIGURATIONS
%----------------------------------------------------------------------------------------

\documentclass[paper=a4, french]{scrartcl} % A4 paper and 11pt font size

\usepackage[utf8]{inputenc} 
\usepackage[T1]{fontenc} % Use 8-bit encoding that has 256 glyphs
\usepackage{fourier} % Use the Adobe Utopia font for the document - comment this line to return to the LaTeX default
\usepackage[francais]{babel} % English language/hyphenation
\usepackage{amsmath,amsfonts,amsthm} % Math packages
\usepackage{envmath}
\usepackage{lipsum} % Used for inserting dummy 'Lorem ipsum' text into the template
\usepackage{colortbl}

\definecolor{bg}{RGB}{235,235,235}

\usepackage{sectsty} % Allows customizing section commands
\allsectionsfont{\centering \normalfont\scshape} % Make all sections centered, the default font and small caps

\usepackage{fancyhdr} % Custom headers and footers
\pagestyle{fancyplain} % Makes all pages in the document conform to the custom headers and footers
\fancyhead{} % No page header - if you want one, create it in the same way as the footers below
\fancyfoot[L]{} % Empty left footer
\fancyfoot[C]{} % Empty center footer
\fancyfoot[R]{\thepage} % Page numbering for right footer
\renewcommand{\headrulewidth}{0pt} % Remove header underlines
\renewcommand{\footrulewidth}{0pt} % Remove footer underlines
\setlength{\headheight}{13.6pt} % Customize the height of the header

\numberwithin{equation}{section} % Number equations within sections (i.e. 1.1, 1.2, 2.1, 2.2 instead of 1, 2, 3, 4)
\numberwithin{figure}{section} % Number figures within sections (i.e. 1.1, 1.2, 2.1, 2.2 instead of 1, 2, 3, 4)
\numberwithin{table}{section} % Number tables within sections (i.e. 1.1, 1.2, 2.1, 2.2 instead of 1, 2, 3, 4)

\setlength\parindent{0pt} % Removes all indentation from paragraphs - comment this line for an assignment with lots of text
\newcommand{\class}[1]{\colorbox{bg}{\textcolor{red}{\usefont{OT1}{cmtt}{m}{n}#1}}}

%----------------------------------------------------------------------------------------
%	TITLE SECTION
%----------------------------------------------------------------------------------------

\newcommand{\horrule}[1]{\rule{\linewidth}{#1}} % Create horizontal rule command with 1 argument of height

\title{	
\normalfont \normalsize 
\textsc{Ecole Polytechnique} \\ [25pt] % Your university, school and/or department name(s)
\horrule{0.5pt} \\[0.4cm] % Thin top horizontal rule
\huge Set multijoueurs \\ % The assignment title
\horrule{2pt} \\[0.5cm] % Thick bottom horizontal rule
}

\author{Zhixing CAO, Yuxiang LI} % Your name

\date{\normalsize\today} % Today's date or a custom date

\begin{document}

\setlength\parindent{12pt}

\maketitle % Print the title

% + - = ! / ( ) [ ] < > | ' :

%----------------------------------------------------------------------------------------
%	Introduction
%----------------------------------------------------------------------------------------

\section{Introduction}

Ce projet consiste à développer une application sur Android en appliquant la programmation concurrente pour réaliser une communication serveur-client. Cette application permet de jouer un jeu appelé \textbf{Set} soit en mode sole, soit un multijoueur. On présentera d'abord la règle du jeu et leur correspondance en terme de Java ou Android. On expliquera ensuite plus en détail la réalisation de tous les modules nécessaires à ce jeu, on verra en quoi la programmation multithread permet d'avoir une meilleure performance de l'application.

%----------------------------------------------------------------------------------------
%	Protocole
%----------------------------------------------------------------------------------------

\section{Principe du jeu et protocole de communication }

%-------------------------------------
\subsection{Comment jouer le jeu ?}

\subsubsection{Présentation}
Le jeu Set est un jeu de cartes constitué de 81 cartes toutes différentes qui se distinguent selon 4 caractéristiques : la couleur, la forme, le type de remplissage et le nombre d'objets. Le but du jeu est de trouver une ensemble de 3 cartes qui ont des caractéristiques soit toutes les trois identiques, soit toutes les trois différentes. Le jeu peut se jouer à un ou plusieurs joueurs, à chaque tour, on dispose 12 cartes sur la table, et le premier joueur qui trouve un bon set de 3 cartes gagne et ces 3 cartes vont être remplacés par 3 autre. S'il n'y a pas de bon set parmi ces 12 cartes, on rajoute 3 cartes supplémentaires en supposant qu'il y ait toujours un set parmi elles.

\subsubsection{Choisir les cartes}
Le choix des cartes se fait l'une après l'autre, les cartes entourées du bleus sont choisis, reclicker sur la carte déjà choisie permet de l'enveler du set. Une fois qu'on a 3 cartes, on soumet directement la solution.

\subsubsection{Score et pénalité}
En mode solo, le set est jugé sur place, tout set correct sera coloré en vert et incorrect en rouge. La couleur disparaît après une petite durée. Une fois que le set est jugé, on ne peut plus le toucher avant la disparition de la couleur. Le set correct sera ensuite remplacé par 3 nouvelles cartes. Un set correct rapporte 10 points et on enlève 2 points pour un faux set.
 
 
%------------------------------------------------
\subsection{Comment communiquer avec l'autre machine ?}

\subsubsection{Serveur}
Le serveur gère tout le déroulement du jeu en suivant le protocle suivant : 
\begin{itemize}
\item Initialiser le jeu : initialiser les cartes
\item Changer les cartes 
\item Attribuer les points suivant l'ordre d'arrivée du message : le deuxième ne reçoit rien
\item Débloquer le jeu si tous les joueurs sont bloqués
\item Faire sortir un joueur après une longue durée de silence
\end{itemize}
	
\subsubsection{Client}
Le côté client est plus simple. 
\begin{itemize}
\item Recevoir les message : changer les cartes, renouveller le score ou débloquer le jeu
\item Juger le set comme dans le mode solo
\item Se bloquer pour un faux set et soumettre le résultat pour un bon
\item Quitter le jeu en prévenant le serveur
\end{itemize}

	
%----------------------------------------------------------------------------------------
%	Description du projet
%----------------------------------------------------------------------------------------

\section{Du côté de Java}
%-------------------------------------

\subsection{Android}

La partie Android se divise principalement en 2 parties : la class \class{CardView} correspondant à View de \textbf{MVC} et \class{MainActivity} jouant à la fois le rôle de Model et de Controller.

\begin{itemize}
\item \class{CardView} et \class{Card}

Bien qu'on n'ait que 81 cartes, on a choisi de les présenter avec une class au lieu d'un entier, ce choix permet d'avoir une plus grande liberté et clarté du code, la méthode \class{hashCode} et le constructeur permettent de créer une bijection entre les deux espaces d'états pour faciliter la communication. Tandis que \class{CardView}, qui implémente la classe \class{View} gère l'affichage des cartes suivant les paramètres (membres de la classe) : carte de score, carte gêlée, carte correcte, carte choisie, etc. En conséquence, les \class{OnClickListener} sont directement associés à ces \class{CardView}, ce qui évite de compliquer le traitement des événements.

\item \class{MainActivity}

Cette classe se charge d'interagir avec l'utilisateur et de distribuer le travail au sein de l'application (communication Socket, calcul, affichage). Il permet d'augmenter ou diminuer le score, changer les couleurs des 'CardView' et remplacer les cartes selon les cartes choisits par l'utilisateur. Plusieurs classes du type \class{Thread} et \class{Runnable} sont définies dans cette partie en rendant l'application multi-tâches.

\end{itemize}
 
%-------------------------------------
\subsection{Serveur}

Le code du serveur est réalisé en-dehors de l'application Android par une nouvelle classe : \class{Server}. Cette classe gère la création d'un nouveau \class{Thread} à chaque nouveau joueur, l'attribution du point, le bon déroulement du jeu. Les différents messages sont listés ci-dessous : 

\begin{itemize}
\item (S)Modifier les cartes : V{\textvisiblespace}$pos_1${\textvisiblespace}$card_1${\textvisiblespace }$pos_2${\textvisiblespace}$card_2${\textvisiblespace}$pos_3${\textvisiblespace}$card_3$
\item (S)Incrémenter le score : S{\textvisiblespace}$card_1${\textvisiblespace}$card_2${\textvisiblespace}$card_3$
\item (S)Faire sortir le joueur : E
\item (C)Demander le redémarrage : R
\item (C)Informer sur l'état de blocage : F
\item (C)Soumettre le résultat : S{\textvisiblespace}$pos_1${\textvisiblespace}$card_1${\textvisiblespace }$pos_2${\textvisiblespace}$card_2${\textvisiblespace}$pos_3${\textvisiblespace}$card_3$
\item (C)Quitter le jeu : E
\end{itemize}


%----------------------------------------------------------------------------------------
%	Détail de réalisation
%----------------------------------------------------------------------------------------

\section{Réalisation}

%-------------------------------------
	\subsubsection{Le tas de carte}
	Pour créer le tas de carte, on veut tirer aléatoirement les 81 entiers entre 0 et 80 qui encode les 81 cartes. Pour faire cela, on a d'abord crée 81 entiers au hasard par la méthode 'random'. Et puis, en gardant les indices de ces 81 entiers, on les trie en ordre croissant. Et comme cela, les indices de ces entiers répartiront aléatoirement. Après en décodant ces entiers, on a bien un tas de carte 'aléatoir'.
	
	\subsubsection{Handler}
%----------------------------------------------------------------------------------------

\end{document}